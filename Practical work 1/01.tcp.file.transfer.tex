\documentclass[a4paper,12pt]{article}
\usepackage[utf8]{inputenc}
\usepackage{graphicx}
\usepackage{listings}
\usepackage{xcolor}
\usepackage{float}
\usepackage{geometry}
\geometry{left=2.5cm, right=2.5cm, top=2cm, bottom=2cm}

\lstset{
    language=C,
    basicstyle=\ttfamily\small,
    keywordstyle=\color{blue}\bfseries,
    stringstyle=\color{red},
    commentstyle=\color{green!50!black},
    numbers=left,
    numberstyle=\tiny,
    stepnumber=1,
    frame=single,
    breaklines=true,
    tabsize=4,
    showstringspaces=false,
    captionpos=b
}

\title{\textbf{Practical Work 1: TCP File Transfer}}
\author{
    Group Name \\ 
    \small (Member 1 Name - ID) \\
    \small (Member 2 Name - ID) \\
    \small (Member 3 Name - ID) \\
    \\
    University of Science and Technology of Hanoi
}
\date{\today}

\begin{document}

\maketitle

\section{Protocol Design}
In this practical work, we designed a simple file transfer protocol based on TCP/IP. The communication flow follows the standard Client-Server model using the POSIX socket API (Linux).

\begin{itemize}
    \item \textbf{Server State:} The server creates a socket, binds it to port \textbf{8080}, and enters the listening state waiting for connections.
    \item \textbf{Connection:} The client creates a socket and requests a connection to the server's IP address.
    \item \textbf{Data Exchange:} 
    \begin{itemize}
        \item The client opens the file \texttt{send\_me.txt} and reads it in chunks (buffer size 1024 bytes).
        \item The client sends these chunks to the server using \texttt{send()}.
        \item The server receives data using \texttt{read()} and writes it to \texttt{received\_file.txt}.
    \end{itemize}
    \item \textbf{Termination:} After the file is completely sent, the client closes the file and the socket. The server detects the end of transmission (return value 0) and closes its resources.
\end{itemize}

\begin{figure}[H]
    \centering
    \caption{TCP File Transfer Protocol Design}
\end{figure}

\section{System Organization}
The system is organized into two separate programs running in the terminal:

\begin{enumerate}
    \item \textbf{Server Application (\texttt{server.c}):} 
    Acts as the passive entity. It runs indefinitely (or until one transfer is done), accepts a connection, and saves the incoming data stream to a file.
    
    \item \textbf{Client Application (\texttt{client.c}):} 
    Acts as the active entity. It initiates the TCP handshake, reads a local file, and streams the data to the server.
\end{enumerate}

\section{Implementation}
Below are the code snippets showing the core logic of the file transfer implemented in C for the Linux environment.

\subsection{Server-side: Receiving Data}
We use the \texttt{read()} system call to receive data from the socket file descriptor.

\begin{lstlisting}[caption={Server loop to receive and save file}]
FILE *fp = fopen("received_file.txt", "wb");
int valread;

// Loop to read from socket until connection is closed
while ((valread = read(new_socket, buffer, BUFFER_SIZE)) > 0) {
    fwrite(buffer, 1, valread, fp);
    memset(buffer, 0, BUFFER_SIZE);
}

fclose(fp);
close(new_socket);
\end{lstlisting}

\subsection{Client-side: Sending Data}
We use standard file I/O to read the file and \texttt{send()} to push data to the network.

\begin{lstlisting}[caption={Client loop to read and send file}]
FILE *fp = fopen("send_me.txt", "rb");
int bytes_read;

// Loop to read from file and send to socket
while ((bytes_read = fread(buffer, 1, BUFFER_SIZE, fp)) > 0) {
    send(sock, buffer, bytes_read, 0);
    memset(buffer, 0, BUFFER_SIZE);
}

fclose(fp);
close(sock);
\end{lstlisting}

\section{Experimental Results}
We successfully compiled and ran the code on a Linux environment (Ubuntu/WSL) using the GCC compiler.
\begin{itemize}
    \item Compiler command: \texttt{gcc server.c -o server} and \texttt{gcc client.c -o client}
    \item Result: The file was transferred successfully without data corruption.
\end{itemize}

\begin{figure}[H]
    \centering
    \includegraphics[width=0.9\textwidth]{screenshot_server.png}
    \caption{Server listening and receiving file on Linux}
\end{figure}

\begin{figure}[H]
    \centering
    \includegraphics[width=0.9\textwidth]{screenshot_client.png}
    \caption{Client connecting and sending file on Linux}
\end{figure}

\section{Roles and Responsibilities}
\begin{table}[H]
\centering
\begin{tabular}{|p{4cm}|p{8cm}|}
\hline
\textbf{Member Name} & \textbf{Task Description} \\ \hline
Member 1 (Leader) & Implementation of Server side, Makefile, and Report consolidation \\ \hline
Member 2 & Implementation of Client side, Testing on Linux environment \\ \hline
Member 3 & Protocol Design, System Architecture Diagram, and Debugging \\ \hline
\end{tabular}
\caption{Group Work Distribution}
\end{table}

\end{document}
