\documentclass[a4paper,12pt]{article}
\usepackage[utf8]{inputenc}
\usepackage{graphicx} 
\usepackage{listings}
\usepackage{xcolor}  
\usepackage{float}    

\lstset{
    language=C,
    basicstyle=\ttfamily\small,
    keywordstyle=\color{blue}\bfseries,
    stringstyle=\color{red},
    commentstyle=\color{green!50!black},
    numbers=left,
    numberstyle=\tiny,
    frame=single,
    breaklines=true
}

\title{\textbf{Practical Work 1: TCP File Transfer}}
\author{Lê Quang Vinh / 23BI14457 \\ University of Science and Technology of Hanoi}
\date{\today}

\begin{document}

\maketitle

\section{Protocol Design}
In this practical work, we designed a simple file transfer protocol based on TCP/IP. The communication flow is as follows:

\begin{itemize}
    \item \textbf{Server:} Creates a socket, binds it to a port (8080), and listens for incoming connections.
    \item \textbf{Client:} Creates a socket and connects to the server's IP address and port.
    \item \textbf{Data Transfer:} Once connected, the client reads a file from its local disk and sends it in chunks (buffer size 1024 bytes) to the server.
    \item \textbf{Termination:} After sending the file, the client closes the connection.
\end{itemize}

\section{System Organization}
The system consists of two main components:
\begin{enumerate}
    \item \textbf{Server Application:} Acts as the passive entity, saving data as \texttt{received\_file.txt}.
    \item \textbf{Client Application:} Acts as the active entity, sending content of \texttt{send\_me.txt}.
\end{enumerate}

\section{Implementation}
Below are the code snippets showing the core logic using Winsock.

\subsection{Server-side: Receiving Data}
\begin{lstlisting}
// Server receiving loop
FILE *fp = fopen("received_file.txt", "wb");
int valread;
while ((valread = recv(new_socket, buffer, BUFFER_SIZE, 0)) > 0) {
    fwrite(buffer, 1, valread, fp);
    memset(buffer, 0, BUFFER_SIZE);
}
fclose(fp);
\end{lstlisting}

\subsection{Client-side: Sending Data}
\begin{lstlisting}
// Client sending loop
FILE *fp = fopen("send_me.txt", "rb");
int bytes_read;
while ((bytes_read = fread(buffer, 1, BUFFER_SIZE, fp)) > 0) {
    send(sock, buffer, bytes_read, 0);
    memset(buffer, 0, BUFFER_SIZE);
}
fclose(fp);
\end{lstlisting}

\section{Experimental Results}
We successfully compiled and ran the code on Windows using MinGW (GCC).

\begin{figure}[H]
    \centering
    \includegraphics[width=1\textwidth]{screenshot_server.png}
    \caption{Server receiving connection}
\end{figure}

\begin{figure}[H]
    \centering
    \includegraphics[width=1\textwidth]{screenshot_client.png}
    \caption{Client sending file}
\end{figure}
% -------------------------------

\section{Roles and Responsibilities}
\begin{table}[H]
\centering
\begin{tabular}{|l|l|}
\hline
\textbf{Member Name} & \textbf{Task} \\ \hline
Your Name & Coding Server side \& Report writing \\ \hline
Member 2 & Coding Client side \& Testing \\ \hline
Member 3 & Protocol Design \& Diagrams \\ \hline
\end{tabular}
\caption{Work distribution}
\end{table}

\end{document}
