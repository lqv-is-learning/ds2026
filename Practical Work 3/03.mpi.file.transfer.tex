\documentclass[a4paper,12pt]{article}
\usepackage[utf8]{inputenc}
\usepackage{graphicx} 
\usepackage{listings}
\usepackage{xcolor}  
\usepackage{float}
\usepackage{hyperref}

% Cấu hình hiển thị code Python
\lstset{
    language=Python,
    basicstyle=\ttfamily\small,
    keywordstyle=\color{blue}\bfseries,
    stringstyle=\color{red},
    commentstyle=\color{green!50!black},
    numbers=left,
    numberstyle=\tiny,
    frame=single,
    breaklines=true,
    showstringspaces=false,
    captionpos=b
}

\title{\textbf{Practical Work 3: MPI File Transfer}}
\author{LÊ QUANG VINH \\ University of Science and Technology of Hanoi}
\date{\today}

\begin{document}

\maketitle

\section{Introduction}
In this practical work, we explore \textbf{Message Passing Interface (MPI)}, a standard for parallel computing. The goal is to transfer a file between two distinct processes (ranks) within a communicator group.

We selected \textbf{OpenMPI} combined with the \textbf{mpi4py} Python library on a \textbf{Linux} environment.
\begin{itemize}
    \item \textbf{Why OpenMPI?} It is a high-performance, open-source MPI implementation that is widely used and highly compatible with Linux systems.
    \item \textbf{Why mpi4py?} It provides a Pythonic interface to MPI, handling object serialization (pickling) automatically, which simplifies the implementation of file transfer logic compared to low-level C/C++ bindings.
\end{itemize}

\section{System Design}
The system relies on the \texttt{MPI.COMM\_WORLD} communicator. We utilize two processes:
\begin{itemize}
    \item \textbf{Rank 0 (Sender):} Reads the file from the disk and sends the binary data using \texttt{comm.send()}.
    \item \textbf{Rank 1 (Receiver):} Waits for incoming data using \texttt{comm.recv()} and writes the received bytes to a new file.
\end{itemize}

\begin{figure}[H]
    \centering
    \begin{verbatim}
    +----------------+                       +----------------+
    |   Process 0    |   (MPI Channel)       |   Process 1    |
    |    (Sender)    | --------------------> |   (Receiver)   |
    | Reads "input"  |      comm.send        | Writes "output"|
    +----------------+                       +----------------+
    \end{verbatim}
    \caption{MPI Point-to-Point Communication Design}
\end{figure}

\section{Implementation}
Below is the core logic for the file transfer using blocking point-to-point communication.

\begin{lstlisting}[caption=mpi\_transfer.py]
from mpi4py import MPI
import os

comm = MPI.COMM_WORLD
rank = comm.Get_rank()
TAG_DATA = 11

def sender():
    filename = "send_me.txt"
    if not os.path.exists(filename):
        print(f"[Rank {rank}] Error: File not found.")
        return
        
    # Read file binary
    with open(filename, "rb") as f:
        data = f.read()
        
    print(f"[Rank {rank}] Sending {len(data)} bytes to Rank 1...")
    # Blocking send to Rank 1
    comm.send(data, dest=1, tag=TAG_DATA)
    print(f"[Rank {rank}] Send complete.")

def receiver():
    print(f"[Rank {rank}] Waiting for data...")
    # Blocking receive from Rank 0
    data = comm.recv(source=0, tag=TAG_DATA)
    
    # Write to new file
    output_name = "received_via_mpi.txt"
    with open(output_name, "wb") as f:
        f.write(data)
    print(f"[Rank {rank}] Received {len(data)} bytes.")
    print(f"[Rank {rank}] Saved as '{output_name}'. Success!")

if rank == 0:
    sender()
elif rank == 1:
    receiver()
\end{lstlisting}

\section{Experimental Results}
The program was executed on Linux using the command:
\begin{verbatim}
mpirun -n 2 python3 mpi_transfer.py
\end{verbatim}

Rank 0 successfully read \texttt{send\_me.txt} (containing 14 bytes) and transmitted it to Rank 1. Rank 1 received the data and saved it as \texttt{received\_via\_mpi.txt}.

\begin{figure}[H]
    \centering
    \includegraphics[width=1\textwidth]{mpi_file_transfer.png}
    \caption{Execution result showing successful transfer between Rank 0 and Rank 1}
\end{figure}

\section{Roles and Responsibilities}
\begin{table}[H]
\centering
\begin{tabular}{|l|l|}
\hline
\textbf{Member Name} & \textbf{Task} \\ \hline
LÊ QUANG VINH & Implementation, MPI Setup (Linux) \& Report \\ \hline
\end{tabular}
\caption{Work distribution}
\end{table}

\end{document}
